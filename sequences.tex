
\section{Sequences}

\subsection{Convergence definition}

A sequence converges to $L$ if  $\forall ϵ>0, ∃N∈\setnaturals$ such that $∀n≥N, \abs{a_n-L} <ϵ$.



\subsection{Monotonous and bounded sequences converge}

\begin{property}{If a sequence $a_n$ is bounded above and is increasing, then it converges}
    \begin{precondition}
        \begin{itemize}
            \item $∃L,∀n \; a_n \leq L$
            \item $∀n \; a_n ≤ a_{n+1}$
        \end{itemize}
    \end{precondition}
    \begin{claim}
        $\liman$ converges to the supremum of  $A=\setdef{a_n}{n ∈ \setnaturals}$.
    \end{claim}
    \begin{Proof}

Since $a_n$ is bounded, then we have that $α = \sup A$ exists by properties of the real numbers.

By definition of supremum, $∀ϵ>0,∃N$ such that:
\begin{equation*}
  α-ϵ<a_N<α
\end{equation*}

 Since $a_n$ is increasing:
 \begin{equation*}
∀n, \; N≤n, \; a_N ≤ a_n
 \end{equation*}
Also by definition:

\begin{equation*}
  ∀n, \; a_n ≤ α
\end{equation*}

Combining these two results, we get $∀n≥N,\;  α-ϵ<a_n<α$. This implies:
\begin{equation*}
  ∀n≥N,\;  α-a_n<ϵ
\end{equation*}
Again, since $a_n≤α$, we have:

\begin{equation*}
  ∀n, α-a_n= \abs{α-a_n}
\end{equation*}
Therefore, given $ϵ>0$, we have $∀n≥N,\;  \abs{α-a_n}<ϵ$, which by definition means:

\begin{equation*}
  \liman = α
\end{equation*}

    \end{Proof}
\end{property}
